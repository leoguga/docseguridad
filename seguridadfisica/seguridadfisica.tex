\label{ch:fisica}
En este cap�tulo se indicar�n las medidas que se usar�n para proteger
los activos de la organizaci�n, de acuerdo con lo indicado en el tema
de <<Seguridad en el entorno>>. Las medidas a implantar se basar�n en
un an�lisis previo de los riesgos existentes.

Se recomienda consultar <<Seguridad f�sica COMO>>, el est�ndar
UNE-ISO/IEC 272002~\cite{une27002} y las normas UNE
71501~\cite{une71501} disponibles en el Campus Virtual. En la norma
UNE 71501-3 se recogen algunos de los tipos de riesgos m�s comunes:
entradas no autorizadas, rayos, robos, incendios, accesos no
autorizados a estaciones de trabajo, etc.

\section{Activos}

Activos de la organizaci�n a nivel f�sico: locales a proteger,
teniendo en cuenta su divisi�n en �reas m�s y menos sensibles.

\section{Seguridad del edificio}

Edificio en general: fluido el�ctrico, l�neas de tel�fono, protecci�n
contra entradas no autorizadas, protecci�n contra incendios, etc.

\section{Seguridad del centro de datos}

Protecci�n del centro de datos en que se alojan los servidores:
emplazamiento, uso de falso techo/suelo, aire acondicionado, control
de acceso y auditor�as, etc. Es posible que la protecci�n a incendios
o alguna de las medidas a nivel de edificio cambie aqu�.

\section{Seguridad del lugar de trabajo}

Protecci�n de la estaci�n de trabajo de cada empleado, evitando
accidentes laborales, entradas no autorizadas, robos, volcado de
l�quidos, etc.