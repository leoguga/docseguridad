\label{ch:informacion}
Este cap�tulo puede dedicarse a los riesgos que afectan a la
informaci�n almacenada y obtenida en la organizaci�n. Para ello, habr�
que describir qu� informaci�n a nivel abstracto trata la organizaci�n
y cu�l es su nivel de sensibilidad.

Habr� que relacionarla con la Ley Org�nica de Protecci�n de Datos,
aunque los documentos de seguridad para la LOPD tendr�n que elaborarse
por separado de este trabajo. Para los documentos de seguridad se
pueden seguir los modelos disponibles en el Campus Virtual.

Este cap�tulo har� referencia a las medidas implantadas en otros
cap�tulos dedicados a la seguridad de las comunicaciones
(cap�tulo~\vref{ch:comunicaciones}), del software y hardware
(cap�tulo~\vref{ch:swhw}) y de la seguridad f�sica
(cap�tulo~\vref{ch:fisica}), entre otros.

\section{Informaci�n recogida}

Descripci�n a alto nivel de la informaci�n que utiliza la organizaci�n
para llevar a cabo sus fines.

\section{An�lisis de riesgos y sensibilidad}


Aqu� se har�a un estudio de lo delicada que es la informaci�n tratada,
y de a qu� riesgos se halla sometida, de acuerdo con las tres bases de
la seguridad (confidencialidad, integridad y disponibilidad).

\section{Conformidad con la legislaci�n vigente}

Se equiparar� la informaci�n antes recogida con los requisitos
impuestos por la Ley Org�nica de Protecci�n de Datos y otras leyes
aplicables, si las hay, y se har� referencia a sus documentos de
seguridad, a elaborar por separado.