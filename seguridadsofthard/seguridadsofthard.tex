\label{ch:swhw}
En este cap�tulo se tratar�n los aspectos relacionados con la
seguridad del software desarrollado y/o utilizado internamente, y la
prevenci�n, detecci�n y diagn�stico del software malicioso. Tambi�n se
tratar�n los aspectos relacionados con el hardware de los activos de
la organizaci�n.

Para los aspectos de desarrollo de software, se puede tomar como gu�a
en la secci�n de desarrollo las recomendaciones de las transparencias
y los informes CWE m�s relevantes, posiblemente integr�ndolos en las
revisiones peri�dicas del c�digo, si se desean implantar.

Habr� que analizar los riesgos que supone el software malicioso, y en
base a ellos plantear las debidas medidas de prevenci�n, detecci�n y
recuperaci�n.

\section{Identificaci�n y autenticaci�n}

Se recogen las distintas medidas a trav�s de las cuales se limitar� el
acceso a los distintos equipos al personal que los necesite para su
trabajo.

\section{Registro de accesos}

En algunos equipos puede ser útil llevar un control de qui�n ha
accedido, desde d�nde, etc.

\section{Protecci�n ante software malicioso}

Medidas para mitigar los riesgos relacionados con el software
malicioso.

\section{Uso aceptable de los equipos}

Se recoger�a cu�l es el uso aceptable de los equipos de la
organizaci�n, dividi�ndolos seg�n su aplicaci�n (servidores,
estaciones de trabajo y port�tiles).

\section{Gesti�n de soportes}

Medidas a la hora de almacenar y desechar los soportes de
almacenamiento.