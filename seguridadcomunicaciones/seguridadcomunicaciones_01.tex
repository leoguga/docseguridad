\label{ch:comunicaciones}
Esta secci�n se dedicar� a describir c�mo se proteger�n las conexiones
internas y externas, los datos que recorran la red interna de la
organizaci�n, y c�mo se evitar�n las fugas de informaci�n hacia el
exterior.

\section{Infraestructura y topolog�a de la red}

Hay que describir la red existente a alto nivel, con el equipamiento
utilizado. Como m�nimo debe describirse la topolog�a a alto nivel, con
sus enrutadores, conmutadores y estaciones de trabajo. Normalmente se
seguir� un cableado estructurado, si la organizaci�n es lo
suficientemente grande.

\paragraf()
La disposici�n de los distintos elementos que contendr�n los armarios de comunicaciones,
Racks, se ubicar�n de la sigueinte manera(ver figura 1):
\begin{itemize}
	\item La unidad de alimentaci�n interrumpida (SAI) ir� en la parte m�s baja del Rack.
	\item Si hiciera falta colocar una bandeja para colocar trasnformadores, inyectores de
	Access Point, Trasceiver, etc... esta debe de ir inmediatamente encima de la SAI
	\item La elecr�nica de red (switch, router, etc...) ir� situada encima de la bandeja para
	tranasformadores si la hubiera, y en caso contrario seguira a la SAI, dejando un espacio de 2 unidades.
	Entre cada dos elementeos (switch-switch � switch-router) ha de ir un pasahilo.
	Los switch se ir�n numerando de abajo arriba
	\item En la parte superior del Rack, en la primera Unidad, se colocara una bandeja-panel para F.O.
	\item Al panel de F.O., separado por su correspondiente pasahilos y dejando 2 Unidades libres, le seguiran los paneles telef�nicos, esto han se der regleteros 110. Los regleeros 110 han de ir separados por un pasahilo. Se enumeraran de arriba abajo
	\item A continuaci�n de los paneles telef�nico, dejando al menos 1 Unidad libre, se colocaran los paneles RJ45, categor�a 5e � superior. Cada panel RJ ir� encapsulado por arriba y por abajo de sus correspondienes pasahilos. Estos paneles se enumeraran de arriba abajo
\end{itemize}


\section{Seguridad de las conexiones al exterior}

Se describir�n las conexiones al exterior (internet, PBX) y c�mo se
proteger�n ante accesos no autorizados. Si se estima necesario, pueden
tomarse medidas para asegurar la disponibilidad de la conexi�n
(enlaces redundantes, por ejemplo).

Normalmente habr� que establecer un per�metro de seguridad para
proteger la red de la organizaci�n de las amenazas exteriores.

\section{Seguridad de la informaci�n transmitida}

Medidas para evitar que la informaci�n transmitida sea bloqueada,
interceptada, modificada o falsamente fabricada.

\section{Seguridad en el teletrabajo}

Si en la organizaci�n se permite el teletrabajo, habr� que tomar las
medidas necesarias para que esto no suponga una amenaza de seguridad.