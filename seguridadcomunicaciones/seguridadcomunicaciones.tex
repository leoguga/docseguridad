\label{ch:comunicaciones}
Esta secci�n se dedicar� a describir c�mo se proteger�n las conexiones
internas y externas, los datos que recorran la red interna de la
organizaci�n, y c�mo se evitar�n las fugas de informaci�n hacia el
exterior.

\section{Infraestructura y topolog�a de la red}

Hay que describir la red existente a alto nivel, con el equipamiento
utilizado. Como m�nimo debe describirse la topolog�a a alto nivel, con
sus enrutadores, conmutadores y estaciones de trabajo. Normalmente se
seguir� un cableado estructurado, si la organizaci�n es lo
suficientemente grande.

\section{Seguridad de las conexiones al exterior}

Se describir�n las conexiones al exterior (internet, PBX) y c�mo se
proteger�n ante accesos no autorizados. Si se estima necesario, pueden
tomarse medidas para asegurar la disponibilidad de la conexi�n
(enlaces redundantes, por ejemplo).

Normalmente habr� que establecer un per�metro de seguridad para
proteger la red de la organizaci�n de las amenazas exteriores.

\section{Seguridad de la informaci�n transmitida}

Medidas para evitar que la informaci�n transmitida sea bloqueada,
interceptada, modificada o falsamente fabricada.

\section{Seguridad en el teletrabajo}

Si en la organizaci�n se permite el teletrabajo, habr� que tomar las
medidas necesarias para que esto no suponga una amenaza de seguridad.