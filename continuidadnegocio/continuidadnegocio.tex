
Este cap�tulo se dedica fundamentalmente a las medidas de recuperaci�n
en caso de un ataque o un desastre.

\section{Introducci�n}

En esta primera secci�n se puede describir qu� partes son
especialmente vitales para el funcionamiento continuado de la
organizaci�n, y que por lo tanto requieren de una mayor inversi�n de
sus recursos.

Habr� que priorizar la recuperaci�n de ciertos activos frente a otros:
servidores frente a estaciones de trabajo, datos sensibles frente a
informaci�n derivada, etc.

\section{Copias de seguridad}

Esta secci�n se dedica a las copias de seguridad, indicando de qu� se
har�n copias, con qu� frecuencia, de qu� tipos, usando qu� soportes y
juegos de copias, etc.

\section{Recuperaci�n de activos}

Planificaci�n para la recuperaci�n de un activo ante un desastre o un
ataque. Puede tratarse de una parte de las infraestructuras, hardware,
software (con su configuraci�n asociada), o datos.

Para software, por ejemplo, lo usual es mantener documentaci�n
actualizada sobre el estado de la configuraci�n de los sistemas, de
forma que ante un ataque al software se pueda reinstalar y volver a
configurar. Esto requiere se�alar qui�n ser� responsable de mantener
dicha documentaci�n al d�a, con qu� periodicidad ser� revisada, y
qui�n llevar� el control de dichas medidas, entre otros aspectos.