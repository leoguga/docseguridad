\documentclass[a4paper,fontsize=11pt,bibliography=totoc]{scrbook}
%
% Codificaci�n
\usepackage[spanish]{babel}
\usepackage[cp1252]{inputenc}
\usepackage{tikz}
\usepackage{amsmath}%
\usepackage{amsfonts}%
\usepackage{amssymb}%
\usepackage{graphicx}
\usepackage{color}
\usepackage{listings}
\usepackage{supertabular}

\usetikzlibrary{trees,positioning,arrows}

% Fuentes
\usepackage{lmodern}
\usepackage[T1]{fontenc}
\usepackage{chapterfolder}

\usepackage{chapterfolder}

% Formato de p�gina
\usepackage[ilines]{scrpage2}

% Otros paquetes
\usepackage{tabularx}
\usepackage{array}
\usepackage{xspace}
\usepackage{varioref}
\usepackage{microtype}

% hyperref casi siempre es mejor cargarlo al final
\usepackage[colorlinks,citecolor=blue]{hyperref}

\setheadsepline{.4pt}

\newcommand{\laorganizacion}{Bodegas San Dionisio, S.A.}

\subject{Universidad de C�diz \\ Seguridad y Competencias Profesionales \\ Curso 2009/2010}
\title{Pol�tica de Seguridad de Tecnolog�as de la Informaci�n de  \laorganizacion}
\author{Leopoldo Jes�s Guti�rrez Galeano\\Lidia Lebr�n Amaya\\Luis Nadal de Mora\\Rafael S�nchez Mart�nez\\Jes�s Soriano Cand�n}
\date{\today}

\makeatletter
\hypersetup{
  pdftitle={Pol�tica de seguridad de \laorganizacion},
  pdfsubject={Seguridad y Competencias Profesionales 09/10},
  pdfauthor={Leopoldo Jes�s Guti�rrez Galeano,Lidia Lebr�n Amaya,Luis Nadal de Mora,Rafael S�nchez Mart�nez,Jes�s Soriano Cand�n},
  pdfkeywords={seguridad, pol�tica, tecnolog�as, informaci�n, inform�tica}
}
\makeatother

\newcommand{\cellcenter}[1]{\multicolumn{1}{c}{#1}}
\newcommand{\thead}[1]{\textbf{\emph{#1}}}
\newcommand{\espaciocambios}{\rule{0cm}{1.5cm}\xspace}
\setlength{\extrarowheight}{4pt}



\begin{document}

% Portada
\begin{titlepage}
  \maketitle
\end{titlepage}

\frontmatter
\pagestyle{empty}

\tableofcontents
\listoffigures
\listoftables

\chapter{Historial de Cambios}

\begin{center}
  \centering
  \begin{tabular}{|m{.3\textwidth}|m{.65\textwidth}|}
    \cellcenter{\thead{Fecha}} & \cellcenter{\thead{Cambios}} \\
    \hline
    \today & Primera versi�n del documento. \\ \hline
    &  \\ \hline
    &  \\ \hline
    &  \\ \hline
    &  \\ \hline
  \end{tabular}
\end{center}

\cfchapter{Compromiso de la direcci�n}{compromisodireccion}{compromisodireccion}

\mainmatter
\pagestyle{scrheadings}

\cfchapter{Introducci�n}{introduccion}{introduccion}


\cfchapter{Pol�tica de seguridad de la informaci�n}{seguridadinformacion}{seguridadinformacion}


\cfchapter{Pol�tica de seguridad f�sica}{seguridadfisica}{seguridadfisica}


\cfchapter{Pol�tica de control del personal}{controlpersonal}{controlpersonal}


\cfchapter{Pol�tica de seguridad en software y hardware}{seguridadsofthard}{seguridadsofthard}


\cfchapter{Pol�tica de seguridad de las comunicaciones}{seguridadcomunicaciones}{seguridadcomunicaciones}


\cfchapter{Pol�tica de continuidad del negocio}{continuidadnegocio}{continuidadnegocio}

\appendix

\cfchapter{Presupuestos de material}{presupuestomaterial}{presupuestomaterial}

\cfchapter{Acuerdo de Confidencialidad}{anexos}{acuerdoconfidencial}
\chapter{Otros anexos}

Se dedicar�n m�s anexos para aquellas listas y materiales que por su
extensi�n rompan el flujo normal del texto de una determinada
pol�tica, o que se consideren que pueden cambiar con m�s frecuencia.

Posibles anexos incluyen:

\begin{itemize}
\item Formularios a utilizar para gestionar las incidencias
\item Normativa y legislaci�n aplicable
\item Procedimientos del responsable de seguridad y/o el comit� de seguridad
\end{itemize}



%%

\backmatter
\bibliographystyle{IEEEtran}

%\bibitem[CopSeg1]{CopSeg1} Pol�tica de Copias de Respaldo
%\\http://www.pecert.gob.pe/archi/planti/plantilla-politica-backup.doc

%\bibitem[CopSeg2]{CopSeg2} Copias de Seguridad, IES Fco. Romero Vargas, CFGS Administraci�n de Sistemas %Inform�ticos
%\\www.iesromerovargas.net/OASIS/SIM/Documentos/copias\_de\_seguridad.pdf
\bibliography{politicaseguridad}

\end{document}
