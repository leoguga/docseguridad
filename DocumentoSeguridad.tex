\documentclass[a4paper,11pt,bibtotoc,noliststotoc]{scrbook}

% Codificación
\usepackage[spanish]{babel}
\usepackage[utf8x]{inputenc}

% Fuentes
\usepackage{lmodern}
\usepackage[T1]{fontenc}
\usepackage{textcomp}

% Formato de página
\usepackage[ilines]{scrpage2}

% Otros paquetes
\usepackage{tabularx}
\usepackage{array}
\usepackage{xspace}
\usepackage{varioref}
\usepackage{microtype}
\usepackage{supertabular}

% hyperref casi siempre es mejor cargarlo al final
\usepackage[colorlinks,citecolor=blue]{hyperref}

\setheadsepline{.4pt}

\newcommand{\laorganizacion}{Bodegas San Dionisio, S.A.}

\subject{Universidad de Cádiz \\ Seguridad y Competencias Profesionales \\ Curso 2010/2011}
\title{Documento de Seguridad de \laorganizacion}
\author{Leopoldo Jesús Gutiérrez Galeano\\Lidia Lebrón Amaya\\Luis Nadal de Mora\\Rafael Sánchez Martínez\\Jesús Soriano Candón}
\date{\today}

\makeatletter
\hypersetup{
  pdftitle={Documento de Seguridad de \laorganizacion},
  pdfsubject={Seguridad y Competencias Profesionales 10/11},
  pdfauthor={Leopoldo Jesús Gutiérrez Galeano, Lidia Lebrón Amaya, Luis Nadal de Mora, Rafael Sánchez Martínez, Jesús Soriano Candón},
  pdfkeywords={seguridad, política, tecnologías, información, informática}
}
\makeatother

\newcommand{\cellcenter}[1]{\multicolumn{1}{c}{#1}}
\newcommand{\thead}[1]{\textbf{\emph{#1}}}
\newcommand{\espaciocambios}{\rule{0cm}{1.5cm}\xspace}
\setlength{\extrarowheight}{4pt}

\begin{document}

% Portada
\begin{titlepage}
  \maketitle
\end{titlepage}

\frontmatter
\pagestyle{empty}

El presente Documento y sus Anexos, redactados en cumplimiento de lo dispuesto en el Reglamento de Medidas de Seguridad (Real Decreto 994/1999 de 11 de Junio), recogen las medidas de índole técnica y organizativas necesarias para garantizar la protección, confidencialidad, integridad y disponibilidad de los recursos afectados por lo dispuesto en el citado Reglamento y en la Ley Orgánica 15/1999 de 13 de diciembre, de Protección de Datos de Carácter Personal.


\tableofcontents
\listoffigures
\listoftables

\chapter{Historial de cambios}

\begin{center}
  \centering
  \begin{tabular}{|m{.3\textwidth}|m{.65\textwidth}|}
    \cellcenter{\thead{Fecha}} & \cellcenter{\thead{Cambios}} \\
    \hline
    \today & Primera versión del documento. \\ \hline
    &  \\ \hline
    &  \\ \hline
    &  \\ \hline
    &  \\ \hline
  \end{tabular}
\end{center}



\mainmatter
\pagestyle{scrheadings}




\chapter{Ámbito de aplicación del documento}

El presente documento será de aplicación a los ficheros que contienen datos de carácter personal que se hallan bajo la responsabilidad del Director de GI/TI, incluyendo los sistemas de información, soportes y equipos empleados para el tratamiento de datos de carácter personal, que deban ser protegidos de acuerdo a lo dispuesto en normativa vigente, las personas que intervienen en el tratamiento y los locales en los que se ubican.

Las medidas de seguridad se clasifican en tres niveles acumulativos (básico, medio y alto) atendiendo a la naturaleza de la información tratada, en relación con la menor o mayor necesidad de garantizar la confidencialidad y la integridad de la información.

\begin{itemize}
\item \emph{Nivel básico}: Se aplicarán a los ficheros con datos de carácter personal.

\item \emph{Nivel medio}: Ficheros que contengan datos relativos a la comisión de infracciones administrativas o penales, Hacienda Pública, (en estos dos casos, deberán ser de titularidad pública), servicios financieros y los que se rijan por el artículo 29 de la LOPD (prestación de servicios de solvencia y crédito).

\item \emph{Nivel alto}: Ficheros que contengan datos de ideología, religión, creencias, origen racial, salud o vida sexual o los recabados para fines policiales sin consentimiento (en este último caso, también deberán ser de titularidad pública).
\end{itemize}

En concreto, los ficheros sujetos a las medidas de seguridad establecidas en este documento, con indicación del nivel de seguridad correspondiente, son los siguientes:
\begin{itemize}
	\item Fichero de personal: Nivel básico.
	\item Fichero de proveedores: Nivel básico.
	\item Fichero de clientes: Nivel básico.
	\item Fichero financiero: Nivel básico.
	\item Fichero de producción: Nivel básico.
	\item Fichero de informes: Nivel básico.
\end{itemize}

En el Anexo A se describen detalladamente cada uno de los ficheros o tratamientos,junto con los aspectos que les afecten de manera particular.





\chapter{Medidas, normas, procedimientos, reglas y estándares encaminados a garantizar los niveles de seguridad exigidos en este documento}

\begin{itemize}
\item Medidas y normas relativas a la identificación y autenticación del personal autorizado a acceder a los datos personales

La identificación y autenticación a la red corporativa está centralizada. Esto quiere decir que cada empleado tiene su nombre de usuario y contraseña que le permite acceder a:

\begin{enumerate}
\item El propio sistema operativo.

\item Los datos organizados en forma de archivos del sistema operativo, como cualquiera producido por herramientas de oficina (Textos, Hojas de cálculo, imágenes, etc.), y que sean accesibles a través del sistema de ficheros del sistema operativo.

\item Correo electrónico corporativo.

\item El acceso a las distintas aplicaciones corporativas puestas a disposición por la companía, cuyo acceso se prohíbe fuera del entorno de seguridad de la red corporativa. El acceso a través de una red privada virtual está autorizado por esta política.
\end{enumerate}

El \textbf{nombre de usuario} se obtiene a partir del nombre del empleado. Por ejemplo: para un empleado llamado Juán Pérez Madrid su nombre de usuario será: jperez, que se compone por las iniciales de su nombre seguido de su primer apellido.

En caso de que haya dos empleados con el mismo nombre se puede adoptar cualquier regla que permita identificarlos, por ejemplo: vimmartinez, vicmmartinez, etc.

Las medidas mínimas de seguridad de \textbf{contraseña} para usuarios del sistema son:
\begin{itemize}
	\item Tamaño mínimo de 8 caracteres.
	\item Utilización de caracteres alfanuméricos.
	\item Caducidad de las contraseñas a los 2 meses.
	\item Máximo de 3 intentos de acceso. En el caso de fallar 3 veces se bloquea el usuario y se debe contactar con el Administrador de Sistemas para resetear el usuario.
\end{itemize}


\item Control de acceso

El personal sólo accederá a aquellos datos y recursos que precise para el desarrollo de sus funciones.

Para solicitar un alta, modificación o baja de las autorizaciones de acceso a los datos, uno de los empleados del puesto de trabajo en cuestión (o el empleado particular, si la solicitud no abarca todos los empleados de un puesto concreto) deberá entregar el impreso de solicitud al Administrador del Sistema, o en su caso el de Red, dependiendo del tipo de fichero (la autoridad en cuestión se especifica en cada sección del Anexo A correspondiente a cada fichero), el cual deberá llevarla a cabo si la considera oportuna, y con la supervisión del Director de GI/TI.

En el Anexo A, se incluye la relación de usuarios actualizada con acceso autorizado a cada sistema de información. Asimismo, se incluye el tipo de acceso autorizado para cada uno de ellos. Esta lista se actualizará cada vez que se producza un cambio (alta, baja o modificación) en las autorizaciones de acceso a los datos, ya sea por parte de un solo empleado o de la totalidad de los empleados de un puesto concreto.


\item Gestión de soportes

Los soportes que contengan datos de carácter personal deben ser etiquetados para permitir su identificación, inventariados y almacenados en un armario ignífugo dedicado expresamente a ello, dentro de la organización de la empresa, en una sala del edificio de oficinas, lugar de acceso restringido al que solo tendrán acceso las personas con autorización que se relacionan a continuación:

El Director de GI/TI, el Responsable de Seguridad, el Jefe de TI y la persona asignada para realizar las copias de seguridad, si no es el propio Jefe de TI, son las personas autorizadas a acceder al lugar destinado al almacenado de soportes que contengan datos de carácter personal, dentro de la organización de la empresa. \\
Para autorizar a nuevos miembros al acceso, o desautorizar a los existentes se consta de un formulario que deberá ser aceptado por los tres cargos mencionados anteriormente (Director de GI/TI, Responsable de Seguridad y Jefe de TI) -siempre que el emisor del formulario no sea uno de ellos; en este caso firmarían los dos restantes-, facilitando así al nuevo miembro, u obligando a la devolución en su caso, de los medios (llaves y/o tarjetas) necesarios para dicho acceso a los soportes en cuestión.

Los soportes informáticos se almacenarán de acuerdo a las siguientes normas:

Las copias diarias se alamacenarán en el lugar mencionado anteriormete. Se utilizarán 6 cintas para este tipo de copias, una para cada día de la semana, ya que el séptimo día se realizará una copia completa, por lo que la diaría no sería necesaria. Las cintas se reutilizarían en cada ciclo de copias, siendo estas reemplazadas por nuevas cada cierto periodo de tiempo que puedan empezar a deteriorarse por su uso. Este periodo de tiempo será estimado por el responsable del mantenimiento de las copias de seguridad.

Al igual que las copias diarias, las semanales se tratarán con el mismo método, constando este ciclo con 4 cintas que se graban semanalmente. Al final de cada mes se realizará una copia completa, por lo que las cintas podrán ser reutilizadas como se ha explicado anteriormente.

La copia completa mensual se tratará de diferente forma, ya que será trasladada a un edificio ageno a la ubicación actual del equipamiento de la empresa, para evitar la pérdida de los datos originales y copias por catástrofes como incendios, terremotos, inundaciones, etc. Estarán en un lugar con las medidas de seguridad adecuadas y seguras de amenzas como las que hemos nombrado, en una caja fuerte o armario a prueba de fuego.

Todas las copias serán almacenadas siguiendo un orden cronológico, tanto las diarias y semanales en su correspondiente sala, como las copias completas mensuales fuera de las instalaciones principales de Bodegas San Dionisio, S.A. Este orden lógico facilitará la búsqueda de alguna determinada copia de una fecha específica.

Todo el formato físico de copias quedaría correctamente etiquetado. Esto significa que dicha etiqueta estaría cifrada mediante un código interno a la organización. Por ejemplo, se podrían usar 7 caracteres alfanuméricos, de forma que el primero se establezca para indicar a qué departamento pertenecen los datos:

\begin{itemize}
\item 1: Departamento Administrativo 
\item 2: Departamento Comercial
\item 3: Departamento de Producción
\item 4: Departamento de GI/TI
\item 5: Departamento Financiero
\end{itemize}

el segundo caracter indicaría qué tipo de backup es:

\begin{itemize}
\item D: Diario
\item S: Semanal
\item M: Mensual
\end{itemize}

y el resto para la fecha en que se ha realizado la copia, también debidamente encriptada, pudiendo usar indistintamente letras o números para referirse al día, mes y año; por ejemplo, el mes de la A (letra 1 del abecedario) hasta la L (letra 12), y el año el número de años que ha trascurrido desde 1990. Así, la etiqueta 5S03J21 correspondería a una copia semanal que contiene datos del departamente financiero, realizada el 03 de octubre de 2011.

La salida de soportes informáticos que contengan datos de carácter personal, fuera de los locales en donde esté ubicado el sistema de información, únicamente puede ser autorizada por el responsable del fichero o aquel en que se hubiera delegado de acuerdo al siguiente procedimiento: \\
La autorización deberá ser aprobada por el Director de GI/TI -o el Jefe de TI si está delegado a hacerlo en su defecto-. Esta deberá indicar el soporte en cuestión, la finalidad y destino, la forma de envío y la persona que autoriza.

En el Anexo C se incluirán los documentos de autorización relativos a la salida de soportes que contengan datos personales. Asimismo, el Anexo D provee el formulario a rellenar cuando se da de alta o baja a un soporte, para el control y gestión de los mismos.


\item Acceso a datos a través de redes de comunicaciones

Las medidas de seguridad exigibles a los accesos a los datos de carácter personal a través de redes de comunicaciones deberán garantizar un nivel de seguridad equivalente al correspondiente a los accesos en modo local.


\item Régimen de trabajo fuera de los locales de la ubicación del fichero

La ejecución de tratamiento de datos de carácter personal fuera de los locales de la ubicación del fichero deberá ser autorizada expresamente por el responsable del fichero y, en todo caso, deberá garantizarse el nivel de seguridad correspondiente al tipo de fichero tratado. Para ello, existirá una o varias personas autorizadas en el centro o local donde se almacenan los soportes que contengan los datos de carácter personal, que cuiden de la seguridad del transporte y el correcto almacenamiento de los mismos en las instalaciones para ello dedicadas. La autorización deberá contener claramente el tipo de soporte a almacenar, las personas autorizadas al tratamiento de los mismos, el destino donde se almacenarán y la persona que autoriza (en este caso el Director de GI/TI -o el Jefe de TI si está delegado a hacerlo en su defecto-). Esta autorización, como se ha mencionado, se realiza para el tratamiento de los soportes, y no autoriza acceso a la información que contienen.


\item Ficheros temporales

Los ficheros temporales deberán cumplir el nivel de seguridad que les corresponda con arreglo a los criterios expresados en el Reglamento de medidas de seguridad, y serán borrados una vez que hayan dejado de ser necesarios para los fines que motivaron su creación.


\item Copias de seguridad

Es obligatorio realizar copias de respaldo de los ficheros automatizados que contengan datos de carácter personal. Los procedimientos establecidos para las copias de respaldo y para su recuperación garantizarán su reconstrucción en el estado en que se encontraban al tiempo de producirse la pérdida o destrucción.

En el Anexo A se detallan los procedimientos de copia y recuperación de respaldo para cada fichero.

\end{itemize}






\chapter{Procedimiento general de información al personal}

Las funciones y obligaciones de cada una de las personas con acceso a los datos de carácter personal y a los sistemas de información están definidas de forma general en el Capítulo siguiente y de forma específica para cada fichero en la parte del Anexo A correspondiente.

Para asegurar que todas las personas conocen las normas de seguridad que afectan al desarrollo de sus funciones, así como las consecuencias del incumplimiento de las mismas, serán informadas de acuerdo con el siguiente procedimiento: \\
La organización consta de información específica para cada puesto de trabajo acerca de las normas de seguridad y las consecuencias de su incumplimiento, que son facilitadas en el momento que se contrata al empleado, estando este obligado a firmar un documento que indica que el mismo ha leído y entiende estas normas. El empleado recibirá una copia de este documento, siendo su responsabilidad el entendimiento y cumplimiento de estas, así como de las consecuencias del incumplimiento de las mismas.





\chapter{Funciones y obligaciones del personal}

\begin{itemize}
\item Funciones y obligaciones de carácter general.

Todo el personal que acceda a los datos de carácter personal está obligado a conocer y observar las medidas, normas, procedimientos, reglas y estándares que afecten a las funciones que desarrolla.

Constituye una obligación del personal notificar al responsable del fichero (de acuerdo al Anexo A), o de seguridad en su caso, las incidencias de seguridad de las que tengan conocimiento respecto a los recursos protegidos, según los procedimientos establecidos en este Documento, y en concreto en su Capítulo V.

Todas las personas deberán guardar el debido secreto y confidencialidad sobre los datos personales que conozcan en el desarrollo de su trabajo.

\item Funciones y obligaciones del Director de GI/TI. 

Como responsable de los ficheros, es el encargado jurídicamente de la seguridad de los mismos y de las medidas establecidas en el presente documento, implantará las medidas de seguridad establecidas en él y adoptará las medidas necesarias para que el personal afectado por este documento conozca las
normas que afecten al desarrollo de sus funciones.

Además, debe asegurar el buen funcionamiento de copias, almacenado y transporte de los soportes físicos que contengan las copias. Para ello, cuenta con:

\begin {itemize}
\item La opción de delegar al Jefe de IT como autorizado al cumplimiento de algunas funciones que se deben llevar a cabo.

\item El apoyo del Administrador del Sistema y de Red, que son los responsables de asignar, modificar o anular los distintos niveles de accesos a los ficheros de datos por parte de los empleados de la organización. En el Anexo A se especifica quién de cada uno de ellos es el responsable del acceso a cada fichero.

\item El Responsable de Seguridad, encargado de coordinar y controlar las medidas definidas en el presente documento.
\end{itemize}

\end{itemize}





\chapter{Procedimiento de notificación, gestión y respuestas ante las incidencias}

Se considerarán como "incidencias de seguridad", entre otras, cualquier incumplimiento de la normativa desarrollada en este Documento de Seguridad, así como a cualquier anomalía que afecte o pueda afectar a la seguridad de los datos de carácter personal.

El procedimiento a seguir para la notificación de incidencias será:

Cuando ocurra una incidencia, el usuario o administrador deberá registrarla en el Libro de Incidencias o comunicarla al Responsable de Seguridad para que a su vez proceda a su registro.

En la notificación se hará constar :
\begin{itemize}
\item Tipo de incidencia
\item Fecha y hora en que se produjo
\item Persona que realiza la notificación
\item Persona a quien se comunica
\item Efectos que puede producir la incidencia
\item Descripción detallada de la misma
\end{itemize}

En el Anexo E se adjunta el impreso de notificación manual que podrá ser utilizado para la notificación de incidencias.

El registro de incidencias se gestionará mediante el Libro de Incidencias mencionado, que mantendrán las incidencias registradas de los 12 últimos meses, con los datos anteriormente listados.

En el Anexo C se incluirán los documentos de autorización por parte del responsable del fichero relativos a la ejecución de procedimientos de recuperación de datos.





\chapter{Procedimientos de revisión}

\begin{itemize}
\item Revisión del Documento de Seguridad.

El responsable de seguridad es la única persona con capacidad para efectuar cambios en el presente documento, siendo su responsabilidad comunicárselo al resto de usuarios autorizados en caso necesario.

El documento deberá mantenerse en todo momento actualizado y deberá ser revisado siempre que se produzcan cambios relevantes en el sistema de información o en la organización del mismo o con mínimo de 3 meses desde la última revisión. Asimismo, deberá adecuarse, en todo momento, a las disposiciones vigentes en materia de seguridad de los datos de carácter personal.


\end{itemize}


\chapter{Consecuencias del incumplimiento del Documento de Seguridad}

El incumplimiento de las obligaciones y medidas de seguridad establecidas en el presente documento por el personal afectado, se sancionará conforme a la normativa de la ley Orgánica 15/1999.






\appendix
\chapter{Aspectos específicos relativos a los diferentes ficheros}

\section{Aspectos relativos al fichero de personal}

Actualizado a: \today

\begin{itemize}
\item Nombre del fichero o tratamiento: Personal.

\item Unidad/es con acceso al fichero o tratamiento: Director de Administración, Jefe de Personal y Administradores de Personal.

\item Identificador y nombre del fichero en el Registro General de Protección de Datos de la Agencia Española de Protección de Datos:

	\begin{itemize}
	\item Identificador: 111111111
	\item Nombre: Personal de \laorganizacion
	\item Descripción: Datos personales del personal de la empresa, así como los datos correspondientes a las nóminas.
	\end{itemize}

\item Nivel de medidas de seguridad a adoptar: básico.

\item Administrador: Administrador del Sistema.

\item Leyes o regulaciones aplicables que afectan al fichero o tratamiento: ley Orgánica 15/1999.

\item Código Tipo Aplicable: No figura.

\item Estructura del fichero principal: 

\begin{itemize}
\item Nombre
\item Primer apellido
\item Segundo apellido
\item Tipo de Documento
\item Número de documento
\item Sexo
\item Fecha de nacimiento
\item Nacionalidad
\item Dirección
\item Ciudad
\item Código Postal
\item Fecha alta
\item Fecha baja
\item Nombre de usuario
\item Contraseña
\item Cargo
\item Correo electrónico
\item Teléfono 1
\item Teléfono 2
\item Tipo de contrato
\item Salario
\item Curriculum
\end{itemize}


\item Información sobre el fichero o tratamiento:
	\begin{itemize}
	\item Finalidad y usos previstos: Gestión del personal de la organización y sus nóminas.
	\item Personas o colectivos sobre los que se pretenda obtener o que resulten obligados a suministrar los datos personales: Trabajadores de la empresa.
	\item Cesiones previstas: Ninguna.
	\item Transferencias Internacionales: No están previstas.
	\item Procedencia de los datos: Personal de la empresa.
	\item Procedimiento de recogida: Encuesta.
	\item Soporte utilizado para la recogida de datos: Informático.
	\end{itemize}

\item Servicio o Unidad ante el que puedan ejercitarse los derechos de acceso, rectificación, cancelación y oposición: Administrador del Sistema.

\item Descripción del sistema de información: <Describir los sistemas de información automatizados o no en los que se realiza el tratamiento de los datos. En el caso de ficheros automatizados, incluir los equipos físicos>.

\item Descripción detallada de las copias de respaldo y de los procedimientos de recuperación:

Las copias de respaldo y procedimientos de recuperación se realizarán de acuerdo al capítulo 7 del documento \emph{Política de Seguridad de Tecnologías de la Información de Bodegas San Dionisio, S.A}, más concretamente de las secciones 7.2.1 (\emph{Tipos de copias y frecuencia}) y 7.3 (\emph{Recuperación de activos}) respectivamente. 

Se citan a continuación, extraido del documento nombrado, las secciones oportunas correspondientes:

\begin{itemize}
\item Tipos de copias y frecuencia

"Se realizarán dos tipos de copias de seguridad, cada uno con un fin distinto: 

\begin{itemize}
\item Diferenciales: Realizaremos copias diferenciales diariamente, de manera que obtendremos una cinta al día con los datos modificados desde la última copia completa que se haya realizado. 

\item Completas: Se realizarán distinto tipos de copias completas:

	\begin{itemize}
	\item Semanales: Cada semana se realizará una copia completa de los datos de esos siete días. Es decir, al cabo de 4 semanas, se habría obtenido una cinta con los nuevos datos generados o modificados de cada semana.
	\item Mensuales: Al finalizar cada més se realizará una copia completa de los datos del sistema.
	\end{itemize}

\end{itemize}

La razón por la que se realiza la copia completa semanal es para evitar que la copia diaria diferencial llegue a ser suficientemente grande, y así el tiempo de realización de la copia no crezca excesivamente. La diferencia de hacer la copia semanal completa y hacer una diferencial más reside en que el siguiente día las copias diferenciales empezarán a hacerse desde esta última copia, por lo que la cantidad de datos que almacenarán no será nunca superior a la de 6 días de modificación o creación de los mismos.

Dichas copias automatizadas se realizarán en un horario en el que no se esté haciendo uso de las instalaciones, para evitar en medida de lo posible que se estén manejando datos al mismo tiempo que se realiza la copia, por lo que sería una opción lógica que se realicen en horario nocturno. Concretamente, se realizarán a las 4:00, a menos que se realice alguna labor de mantenimiento nocturna puntual a dicha hora; en este caso, el responsable de la programación de las copias de seguridad asignará una nueva hora de acuerdo al horario de mantenimiento estimado.

Para mayor seguridad y confidencialidad, los backups se realizarán cifrando la información, de manera que si fuese necesario una recuperación de los datos, el mismo software de cifrado desencriptaría los datos para su correcta restauración en el sistema, sea este o no el encargado de realizar o leer las copias de seguridad con las que se trabajen.

Por otra parte, el software encargado de la realización de las copias verificaría, una vez acabado el proceso, que el backup se ha realizado correctamente, sin ningún tipo de error, obteniéndose una copia fiable de los datos correspondientes."

\item Recuperación de activos (\emph{Datos})

"Si los daños sufridos han sido sobre los ficheros de datos, se podrían tomar varias acciones dependiendo del daño:

	\begin{itemize}
	\item Si han sido los datos actuales del sistema los que han sufrido el ataque, se tendría que llevar a cabo una restauración o recuperación de los mismos usando para ello las copias de seguridad que se han ido llevando a cabo regularmente, de acuerdo a lo redactado anteriormente. Si el ataque se ha dado durante el último mes, la recuperación de datos se podrá realizar de manera agilizada, ya que se tendría acceso casi inmediato a los datos, que se encontrarían en la sala destinada a ubicar estas copias, en el edificio de oficinas. Por lo que bastaría con recuperar la copia deseada y volcar el contenido en el sistema, recuperando la pérdida o modificación de datos que se ha sufrido.
	\item En cambio, si los datos que han sido objeto del ataque son los backups, se habría de actuar de la manera opuesta. Es decir, si se han perdido o deteriorado las copias de seguridad, la recuperación de estas se haría simplemente realizando nuevas copias de los datos del sistema. Las nuevas copias sustituirían a las que han sufrido los daños, quedando almacenadas en el lugar correspondiente y sin afectar esto a la continua creación de copias de seguridad que debe seguir desarrollándose diariamente.
	\item Si, desafortunadamente, tanto los datos actual del sistema como las copias de seguridad sufren daños, estaríamos ante un problema más serio. En este caso, la recuperación de los datos sería prácticamente imposible, a no ser que se haya perdido por borrado de memoria y se pueda usar un software de recuperación de datos eliminados. En otro caso, como incendio o inundación por ejemplo, no quedaría más remedio que recopilar de nuevo todos los datos perdidos mediante las mismas fuentes que se obtuvieron originalmente. Por este motivo se almacenan las copias mensuales completas en otro lugar físico, minimizando la posibilidad de que se pierdan los datos originales y las copias de ellos. Sería más probable que esto ocurriera con las copias diarias y semanales, pero la recuperación de estos datos sería más asequible que hacerlo de todos los datos de la empresa, ya que serían los datos modificados o creados dentro del último mes a lo sumo."
	\end{itemize}

\end{itemize}


\item Información sobre conexión con otros sistemas: Guarda relación con el fichero financiero, el cual contabiliza, entre otras cosas, el pago de nóminas a los trabajadores de la empresa.

\item Funciones del personal con acceso a los datos personales: Director de Administración, Jefe de Personal y Administradores de Personal.

\item Descripción de los procedimientos de control de acceso e identificación: Cada usuario con autorización de acceso al fichero deberá identificarse en el sistema con el nombre de usuario y contraseña asignados como se ha mencionado en el capítulo 2 de este mismo documento.

\item Relación actualizada de usuarios con acceso autorizado: <Relacionar todos los usuarios que acceden al fichero, con especificación del tipo o grupo de usuarios al que pertenecen, su clave de identificación, nombre y apellidos, unidad, fecha de alta y fecha de baja>.

<Si la relación se mantiene de forma informatizada, indicar aquí cual es el sistema utilizado y la forma de obtener el listado. No obstante, siempre que sea posible, es conveniente imprimir la relación de usuarios y adjuntarla periódicamente a este Anexo>.

\item Terceros que acceden a los datos para la prestación de un servicio: Ninguno.

\item Relación de actualizaciones de este Anexo: 

	\begin{itemize}
	\item Creación del anexo: \today
	\end{itemize}

\end{itemize}







\section{Aspectos relativos al fichero de proveedores}


Actualizado a: \today

\begin{itemize}
\item Nombre del fichero o tratamiento: Proveedores.

\item Unidad/es con acceso al fichero o tratamiento: Director de Administración, Jefe de Logística, Inspector de Mantenimiento y Administradores de Logística.

\item Identificador y nombre del fichero en el Registro General de Protección de Datos de la Agencia Española de Protección de Datos: 
	\begin{itemize}
	\item Identificador: 222222222
	\item Nombre: Proveedores de \laorganizacion
	\item Descripción: Datos de los proveedores, la información sobre la logística de la empresa y las facturas de las materias primas adquiridas y de su transporte, así como los albaranes.
	\end{itemize}

\item Nivel de medidas de seguridad a adoptar: básico.

\item Administrador: Administrador del Sistema.

\item Leyes o regulaciones aplicables que afectan al fichero o tratamiento: ley Orgánica 15/1999.

\item Código Tipo Aplicable: No figura.

\item Estructura del fichero principal: <Incluir los tipos de datos personales incluidos, con especificación de los que, por su naturaleza, afectan a la diferente calificación del nivel de medidas de seguridad a adoptar, según lo indicado en el artículo 4 del Reglamento de Seguridad>.

\item Información sobre el fichero o tratamiento
	\begin{itemize}
	\item Finalidad y usos previstos: Gestión de proovedores.
	\item Personas o colectivos sobre los que se pretenda obtener o que resulten obligados a suministrar los datos personales: Proveedores que tengan relación con la organización.
	\item Cesiones previstas: Ninguna.
	\item Transferencias Internacionales: No están previstas.
	\item Procedencia de los datos: Proveedores.
	\item Procedimiento de recogida: Formulario en papel.
	\item Soporte utilizado para la recogida de datos: Papel inicialmente, con la informatización de los datos a posteriori.
	\end{itemize}

\item Servicio o Unidad ante el que puedan ejercitarse los derechos de acceso, rectificación, cancelación y oposición: Administrador del Sistema.

\item Descripción del sistema de información: <Describir los sistemas de información automatizados o no en los que se realiza el tratamiento de los datos. En el caso de ficheros automatizados, incluir los equipos físicos>.

\item Descripción detallada de las copias de respaldo y de los procedimientos de recuperación:

Las copias de respaldo y procedimientos de recuperación se realizarán de acuerdo al capítulo 7 del documento \emph{Política de Seguridad de Tecnologías de la Información de Bodegas San Dionisio, S.A}, más concretamente de las secciones 7.2.1 (\emph{Tipos de copias y frecuencia}) y 7.3 (\emph{Recuperación de activos}) respectivamente. Ambas secciones citadas en este mismo punto de la sección A.1 del presente apéndice.

\item Información sobre conexión con otros sistemas: <Describir las posibles relaciones con otros ficheros del mismo responsable>.

\item Funciones del personal con acceso a los datos personales: Director de Administración, Jefe de Logística, Inspector de Mantenimiento y Administradores de Logística.

\item Descripción de los procedimientos de control de acceso e identificación: Cada usuario con autorización de acceso al fichero deberá identificarse en el sistema con el nombre de usuario y contraseña asignados como se ha mencionado en el capítulo 2 de este mismo documento.

\item Relación actualizada de usuarios con acceso autorizado: <Relacionar todos los usuarios que acceden al fichero, con especificación del tipo o grupo de usuarios al que pertenecen, su clave de identificación, nombre y apellidos, unidad, fecha de alta y fecha de baja>.

<Si la relación se mantiene de forma informatizada, indicar aquí cual es el sistema utilizado y la forma de obtener el listado. No obstante, siempre que sea posible, es conveniente imprimir la relación de usuarios y adjuntarla periódicamente a este Anexo>.

\item Terceros que acceden a los datos para la prestación de un servicio: Ninguno.

\item Relación de actualizaciones de este Anexo: 

	\begin{itemize}
	\item Creación del anexo: \today
	\end{itemize}

\end{itemize}






\section{Aspectos relativos al fichero de clientes}


Actualizado a: \today

\begin{itemize}
\item Nombre del fichero o tratamiento: Clientes.

\item Unidad/es con acceso al fichero o tratamiento: Director Comercial, Jefe de Ventas, Jefe de Marketing, Comerciales de Marketing, Jefe de Productos Terminados, Administradores de Venta, Comerciales de Venta y Administradores de Terminados.

\item Identificador y nombre del fichero en el Registro General de Protección de Datos de la Agencia Española de Protección de Datos:
	\begin{itemize}
	\item Identificador: 333333333
	\item Nombre: Clientes de \laorganizacion
	\item Descripción: Datos de los clientes, la gestión de pedidos, información sobre las ventas, la facturación y los albaranes correspondientes al transporte de mercancías.
	\end{itemize}

\item Nivel de medidas de seguridad a adoptar: básico.

\item Administrador: Administrador del Sistema.

\item Leyes o regulaciones aplicables que afectan al fichero o tratamiento: ley Orgánica 15/1999.

\item Código Tipo Aplicable: No figura.

\item Estructura del fichero principal: <Incluir los tipos de datos personales incluidos, con especificación de los que, por su naturaleza, afectan a la diferente calificación del nivel de medidas de seguridad a adoptar, según lo indicado en el artículo 4 del Reglamento de Seguridad>.

\item Información sobre el fichero o tratamiento
	\begin{itemize}
	\item Finalidad y usos previstos: Gestión de clientes.
	\item Personas o colectivos sobre los que se pretenda obtener o que resulten obligados a suministrar los datos personales: Clientes de la empresa.
	\item Cesiones previstas: Ninguna.
	\item Transferencias Internacionales: No están previstas.
	\item Procedencia de los datos: Clientes.
	\item Procedimiento de recogida: Encuesta.
	\item Soporte utilizado para la recogida de datos: Informático.
	\end{itemize}

\item Servicio o Unidad ante el que puedan ejercitarse los derechos de acceso, rectificación, cancelación y oposición: Administrador del Sistema.

\item Descripción del sistema de información: <Describir los sistemas de información automatizados o no en los que se realiza el tratamiento de los datos. En el caso de ficheros automatizados, incluir los equipos físicos>.

\item Descripción detallada de las copias de respaldo y de los procedimientos de recuperación:

Las copias de respaldo y procedimientos de recuperación se realizarán de acuerdo al capítulo 7 del documento \emph{Política de Seguridad de Tecnologías de la Información de Bodegas San Dionisio, S.A}, más concretamente de las secciones 7.2.1 (\emph{Tipos de copias y frecuencia}) y 7.3 (\emph{Recuperación de activos}) respectivamente. Ambas secciones citadas en este mismo punto de la sección A.1 del presente apéndice.

\item Información sobre conexión con otros sistemas: <Describir las posibles relaciones con otros ficheros del mismo responsable>.

\item Funciones del personal con acceso a los datos personales: Director Comercial, Jefe de Ventas, Jefe de Marketing, Comerciales de Marketing, Jefe de Productos Terminados, Administradores de Venta, Comerciales de Venta y Administradores de Terminados.

\item Descripción de los procedimientos de control de acceso e identificación: Cada usuario con autorización de acceso al fichero deberá identificarse en el sistema con el nombre de usuario y contraseña asignados como se ha mencionado en el capítulo 2 de este mismo documento.

\item Relación actualizada de usuarios con acceso autorizado: <Relacionar todos los usuarios que acceden al fichero, con especificación del tipo o grupo de usuarios al que pertenecen, su clave de identificación, nombre y apellidos, unidad, fecha de alta y fecha de baja>.

<Si la relación se mantiene de forma informatizada, indicar aquí cual es el sistema utilizado y la forma de obtener el listado. No obstante, siempre que sea posible, es conveniente imprimir la relación de usuarios y adjuntarla periódicamente a este Anexo>.

\item Terceros que acceden a los datos para la prestación de un servicio: Ninguno.


\item Relación de actualizaciones de este Anexo: 

	\begin{itemize}
	\item Creación del anexo: \today
	\end{itemize}

\end{itemize}







\section{Aspectos relativos al fichero financiero}


Actualizado a: \today

\begin{itemize}
\item Nombre del fichero o tratamiento: Financias.

\item Unidad/es con acceso al fichero o tratamiento: Director Financiero, Jefe Contabilidad, Comptroller y Contables.

\item Identificador y nombre del fichero en el Registro General de Protección de Datos de la Agencia Española de Protección de Datos: 
	\begin{itemize}
	\item Identificador: 444444444
	\item Nombre: Financias de \laorganizacion
	\item Descripción: Información sobre los cobros, pagos, la contabilidad de la empresa, información sobre el IVA y los balances contables.
	\end{itemize}

\item Nivel de medidas de seguridad a adoptar: básico.

\item Administrador: Administrador del Sistema.

\item Leyes o regulaciones aplicables que afectan al fichero o tratamiento: ley Orgánica 15/1999.

\item Código Tipo Aplicable: No figura.

\item Estructura del fichero principal: <Incluir los tipos de datos personales incluidos, con especificación de los que, por su naturaleza, afectan a la diferente calificación del nivel de medidas de seguridad a adoptar, según lo indicado en el artículo 4 del Reglamento de Seguridad>.

\item Información sobre el fichero o tratamiento
	\begin{itemize}
	\item Finalidad y usos previstos: Gestión de las financias de la empresa.
	\item Personas o colectivos sobre los que se pretenda obtener o que resulten obligados a suministrar los datos personales: Ninguno.
	\item Cesiones previstas: Ninguna.
	\item Transferencias Internacionales: No están previstas.
	\item Procedencia de los datos: Ventas, pagos, etc.
	\item Procedimiento de recogida: Datos informatizados mediante el personal de la empresa durante la gestión de la misma.
	\item Soporte utilizado para la recogida de datos: Informático.
	\end{itemize}

\item Servicio o Unidad ante el que puedan ejercitarse los derechos de acceso, rectificación, cancelación y oposición: Administrador del Sistema.

\item Descripción del sistema de información: <Describir los sistemas de información automatizados o no en los que se realiza el tratamiento de los datos. En el caso de ficheros automatizados, incluir los equipos físicos>.

\item Descripción detallada de las copias de respaldo y de los procedimientos de recuperación:

Las copias de respaldo y procedimientos de recuperación se realizarán de acuerdo al capítulo 7 del documento \emph{Política de Seguridad de Tecnologías de la Información de Bodegas San Dionisio, S.A}, más concretamente de las secciones 7.2.1 (\emph{Tipos de copias y frecuencia}) y 7.3 (\emph{Recuperación de activos}) respectivamente. Ambas secciones citadas en este mismo punto de la sección A.1 del presente apéndice.

\item Información sobre conexión con otros sistemas: <Describir las posibles relaciones con otros ficheros del mismo responsable>.

\item Funciones del personal con acceso a los datos personales: Director Financiero, Jefe Contabilidad, Comptroller y Contables.

\item Descripción de los procedimientos de control de acceso e identificación: Cada usuario con autorización de acceso al fichero deberá identificarse en el sistema con el nombre de usuario y contraseña asignados como se ha mencionado en el capítulo 2 de este mismo documento.

\item Relación actualizada de usuarios con acceso autorizado: <Relacionar todos los usuarios que acceden al fichero, con especificación del tipo o grupo de usuarios al que pertenecen, su clave de identificación, nombre y apellidos, unidad, fecha de alta y fecha de baja>.

<Si la relación se mantiene de forma informatizada, indicar aquí cual es el sistema utilizado y la forma de obtener el listado. No obstante, siempre que sea posible, es conveniente imprimir la relación de usuarios y adjuntarla periódicamente a este Anexo>.

\item Terceros que acceden a los datos para la prestación de un servicio: Ninguno.

\item Relación de actualizaciones de este Anexo: 

	\begin{itemize}
	\item Creación del anexo: \today
	\end{itemize}

\end{itemize}








\section{Aspectos relativos al fichero de producción}


Actualizado a: \today

\begin{itemize}
\item Nombre del fichero o tratamiento: Producción.

\item Unidad/es con acceso al fichero o tratamiento: Director de Producción, Jefes del Departamento de Producción, Coordinador de Producción, Inspector de Calidad, Técnicos de Laboratorio, Capataces y Operarios.

\item Identificador y nombre del fichero en el Registro General de Protección de Datos de la Agencia Española de Protección de Datos: 
	\begin{itemize}
	\item Identificador: 555555555
	\item Nombre: Producción de \laorganizacion
	\item Descripción: Información sobre la producción de la empresa y sobre las inspecciones de calidad.
	\end{itemize}

\item Nivel de medidas de seguridad a adoptar: básico.

\item Administrador: Administrador de Red.

\item Leyes o regulaciones aplicables que afectan al fichero o tratamiento: ley Orgánica 15/1999.

\item Código Tipo Aplicable: No figura.

\item Estructura del fichero principal: <Incluir los tipos de datos personales incluidos, con especificación de los que, por su naturaleza, afectan a la diferente calificación del nivel de medidas de seguridad a adoptar, según lo indicado en el artículo 4 del Reglamento de Seguridad>.

\item Información sobre el fichero o tratamiento
	\begin{itemize}
	\item Finalidad y usos previstos: Gestión de la producción de la empresa y las inspecciones de calidad.
	\item Personas o colectivos sobre los que se pretenda obtener o que resulten obligados a suministrar los datos personales: Ninguno.
	\item Cesiones previstas: Ninguna.
	\item Transferencias Internacionales: No están previstas.
	\item Procedencia de los datos: Departamentos de Producción.
	\item Procedimiento de recogida: Informatización de los datos durante la getión de producción.
	\item Soporte utilizado para la recogida de datos: Informático.
	\end{itemize}

\item Servicio o Unidad ante el que puedan ejercitarse los derechos de acceso, rectificación, cancelación y oposición: Administrador de Red.

\item Descripción del sistema de información: <Describir los sistemas de información automatizados o no en los que se realiza el tratamiento de los datos. En el caso de ficheros automatizados, incluir los equipos físicos>.

\item Descripción detallada de las copias de respaldo y de los procedimientos de recuperación:

Las copias de respaldo y procedimientos de recuperación se realizarán de acuerdo al capítulo 7 del documento \emph{Política de Seguridad de Tecnologías de la Información de Bodegas San Dionisio, S.A}, más concretamente de las secciones 7.2.1 (\emph{Tipos de copias y frecuencia}) y 7.3 (\emph{Recuperación de activos}) respectivamente. Ambas secciones citadas en este mismo punto de la sección A.1 del presente apéndice.

\item Información sobre conexión con otros sistemas: <Describir las posibles relaciones con otros ficheros del mismo responsable>.

\item Funciones del personal con acceso a los datos personales: Director de Producción, Jefes del Departamento de Producción, Coordinador de Producción, Inspector de Calidad, Técnicos de Laboratorio, Capataces y Operarios.

\item Descripción de los procedimientos de control de acceso e identificación: Cada usuario con autorización de acceso al fichero deberá identificarse en el sistema con el nombre de usuario y contraseña asignados como se ha mencionado en el capítulo 2 de este mismo documento.

\item Relación actualizada de usuarios con acceso autorizado: <Relacionar todos los usuarios que acceden al fichero, con especificación del tipo o grupo de usuarios al que pertenecen, su clave de identificación, nombre y apellidos, unidad, fecha de alta y fecha de baja>.

<Si la relación se mantiene de forma informatizada, indicar aquí cual es el sistema utilizado y la forma de obtener el listado. No obstante, siempre que sea posible, es conveniente imprimir la relación de usuarios y adjuntarla periódicamente a este Anexo>.

\item Terceros que acceden a los datos para la prestación de un servicio: Ninguno.


\item Relación de actualizaciones de este Anexo: 

	\begin{itemize}
	\item Creación del anexo: \today
	\end{itemize}

\end{itemize}








\section{Aspectos relativos al fichero de informes}


Actualizado a: \today

\begin{itemize}
\item Nombre del fichero o tratamiento: Informes.

\item Unidad/es con acceso al fichero o tratamiento: Directores de la empresa y miembros de GI.

\item Identificador y nombre del fichero en el Registro General de Protección de Datos de la Agencia Española de Protección de Datos:
	\begin{itemize}
	\item Identificador: 666666666
	\item Nombre: Iformes de \laorganizacion
	\item Descripción: Informes de dirección, que se pueden acabar representando como dashboards, etc.
	\end{itemize}

\item Nivel de medidas de seguridad a adoptar: básico.

\item Administrador: Administrador de Red.

\item Leyes o regulaciones aplicables que afectan al fichero o tratamiento: ley Orgánica 15/1999.

\item Código Tipo Aplicable: No figura.

\item Estructura del fichero principal: <Incluir los tipos de datos personales incluidos, con especificación de los que, por su naturaleza, afectan a la diferente calificación del nivel de medidas de seguridad a adoptar, según lo indicado en el artículo 4 del Reglamento de Seguridad>.

\item Información sobre el fichero o tratamiento
	\begin{itemize}
	\item Finalidad y usos previstos: Gestión de informes de dirección.
	\item Personas o colectivos sobre los que se pretenda obtener o que resulten obligados a suministrar los datos personales: Ninguno.
	\item Cesiones previstas: Ninguna.
	\item Transferencias Internacionales: No están previstas.
	\item Procedencia de los datos: Directores de la empresa.
	\item Procedimiento de recogida: Formularios de informes.
	\item Soporte utilizado para la recogida de datos: Informático.
	\end{itemize}

\item Servicio o Unidad ante el que puedan ejercitarse los derechos de acceso, rectificación, cancelación y oposición: Administrador de Red.

\item Descripción del sistema de información: <Describir los sistemas de información automatizados o no en los que se realiza el tratamiento de los datos. En el caso de ficheros automatizados, incluir los equipos físicos>.

\item Descripción detallada de las copias de respaldo y de los procedimientos de recuperación:

Las copias de respaldo y procedimientos de recuperación se realizarán de acuerdo al capítulo 7 del documento \emph{Política de Seguridad de Tecnologías de la Información de Bodegas San Dionisio, S.A}, más concretamente de las secciones 7.2.1 (\emph{Tipos de copias y frecuencia}) y 7.3 (\emph{Recuperación de activos}) respectivamente. Ambas secciones citadas en este mismo punto de la sección A.1 del presente apéndice.

\item Información sobre conexión con otros sistemas: <Describir las posibles relaciones con otros ficheros del mismo responsable>.

\item Funciones del personal con acceso a los datos personales: Directores de departamentos y miembros de GI.

\item Descripción de los procedimientos de control de acceso e identificación: Cada usuario con autorización de acceso al fichero deberá identificarse en el sistema con el nombre de usuario y contraseña asignados como se ha mencionado en el capítulo 2 de este mismo documento.

\item Relación actualizada de usuarios con acceso autorizado: <Relacionar todos los usuarios que acceden al fichero, con especificación del tipo o grupo de usuarios al que pertenecen, su clave de identificación, nombre y apellidos, unidad, fecha de alta y fecha de baja>.

<Si la relación se mantiene de forma informatizada, indicar aquí cual es el sistema utilizado y la forma de obtener el listado. No obstante, siempre que sea posible, es conveniente imprimir la relación de usuarios y adjuntarla periódicamente a este Anexo>.

\item Terceros que acceden a los datos para la prestación de un servicio: Ninguno.

\item Relación de actualizaciones de este Anexo: 

	\begin{itemize}
	\item Creación del anexo: \today
	\end{itemize}

\end{itemize}










\chapter{Nombramientos}

<Adjuntar original o copia de los nombramientos que afecten a los diferentes perfiles
incluidos en este documento, como el del responsable de seguridad>







\chapter{Autorizaciones firmadas para la salida o recuperación de datos}


\vspace{-0.3cm}
\begin{center}
\begin{supertabular}{|p{14cm}|}
	\hline
	\hbox to14.4cm{\hss \textbf{AUTORIZACIÓN DE SALIDA O RECUPERACIÓN DE DATOS}\hss}\\
	\hline
\end{supertabular}
\end{center}

\hfill Fecha: \_\_\_\_\_\_\_\_\_\_\_\_\_\_\_\_\_

\vspace{-0.6cm}
\begin{center}
\begin{supertabular}{|p{3cm}|p{11cm}|}
	\hline
	\multicolumn {2}{|c|}{\textbf{SOPORTE}}\\
	\hline
	Identificación &  \\
	\hline
	Contenido & \\
	\hline
	Fichero de donde proceden los datos & \\
	\hline
	Fecha de creación & \\
	\hline
\end{supertabular}
\end{center}


\begin{center}
\begin{supertabular}{|p{3cm}|p{11cm}|}
	\hline
	\multicolumn {2}{|c|}{\textbf{FINALIDAD Y DESTINO}}\\
	\hline
	Finalidad &  \\
	\hline
	Destino & \\
	\hline
	Destinatario & \\
	\hline
\end{supertabular}
\end{center}



\begin{center}
\begin{supertabular}{|p{3cm}|p{11cm}|}
	\hline
	\multicolumn {2}{|c|}{\textbf{FORMA DE ENVÍO}}\\
	\hline
	Medio de envío &  \\
	\hline
	Remitente & \\
	\hline
	Precauciones para el transporte & \\
	\hline
\end{supertabular}
\end{center}



\bottomcaption{Autorización de salida o recuperación de datos}
\begin{center}
\begin{supertabular}{|p{3cm}|p{11cm}|}
	\hline
	\multicolumn {2}{|c|}{\textbf{AUTORIZACIÓN}}\\
	\hline
	Persona que autoriza &  \\
	\hline
	Cargo / Puesto & \\
	\hline
	Observaciones & \\
	\hline
	Firma & \\
	\hline
\end{supertabular}
\end{center}







\chapter{Inventario de soportes}


<Si el inventario de soportes se gestiona de forma no automatizada recoger en este
anexo la información al efecto, según lo indicado en el Capítulo II, punto “Gestión de
soportes” de este documento. Los soportes deberán permitir identificar el tipo de
información, que contienen, ser inventariados y almacenarse en un lugar con acceso
restringido al personal autorizado para ello en este documento >







\chapter{Registro de incidencias}

\bottomcaption{Impreso de notificación de incidencias}
\begin{center}
\begin{supertabular}{|p{17cm}|}
	\hline
	\large \textbf{Incidencia nº:} \_\_\_\_\_\_\_\_ \small (A ser rellenado por el responsable de seguridad)  \\
	\hline
	\large \textbf{Fecha de notificación: \_\_\_\_\_\_\_\_\_\_} \\
	\hline
	\large \textbf{Tipo de incidencia:} \small (Anotar todos los detalles de interés de la incidencia)  \\
	\\
	\\
	\hline
	\large \textbf{Descripción detallada de la incidencia:}  \\
	\\
	\\
	\\
	\hline
	\large \textbf{Fecha y hora en que se produjo la incidencia:}  \\
	\\
	\hline
	\large \textbf{Persona(s) que realiza(n) la notificación:} \small (Especificar sin son usuarios o no del fichero)  \\
	\\
	\\
	\\
	\hline
	\large \textbf{Persona(s) a quien(es) se comunica:}  \\
	\\
	\\
	\hline
	\large \textbf{Efectos que puede producir:} \small (En caso de no subsanación o incluso independientemente de ella)  \\
	\\
	\\
	\hline
	\textbf{Persona que realiza la comunicación:}  \\
	\\
	\textbf{Fdo.:} \_\_\_\_\_\_\_\_\_\_\_\_\_\_\_\_ \\
	\hline
\end{supertabular}
\end{center}




%%

\backmatter
\bibliographystyle{hispa}
\bibliography{bibliografia}

\end{document}

%%% Local Variables: 
%%% mode: latex
%%% TeX-master: t
%%% End: 
